\Introduction  
    Система конвейерной обработки -- это система, 
    основаная на разделении подлежащей выполнению задачи на более мелкие части,
    называемые ступенями, и выделении для каждой из них отдельного блока аппаратуры или потока.
    При этом конвейеризацию можно использовать для совмещения этапов выполнения разных команд. 
    Производительность при этом возрастает благодаря тому,
    что одновременно на различных ступенях конвейера выполняются несколько команд.
    Но не каждую задачу можно разделить на несколько ступеней,
    организовав передачу данных от одного этапа к следующему.

    % Конвейерная обработка такого рода широко применяется во всех современных быстродействующих процессорах.
    
    Целью данной лабораторной работы является реализация системы конвейерной обработки.

    Задачи данной лабораторной работы:
    \begin{enumerate}
        \item описать алгоритмы конвейерной обработки;
        \item реализовать алгоритмы конвейерной обработки;
        \item провести замеры процессорного времени работы.
    \end{enumerate}

\newpage