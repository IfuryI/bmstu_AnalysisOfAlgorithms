\Conclusion
    В ходе выполнения данной лабораторной работы был изучен
    муравьиный алгоритм и приобретены навыки параметризации
    методов на его примере. 
    Реализованы следующие задачи:
    \begin{itemize}
        \item реализованы два алгоритма решения задачи коммивояжера;
        \item замерено время выполнения алгоритмов;
        \item проведено исследование уравьиного алгоритма от параметров.
    \end{itemize}
    
    Использовать муравьиный алгоритм для решения задачи коммивояжера выгодно (с точки зрения времени выполнения),
    в сравнении с алгоритмом полного перебора, в случае если в анализируемом графе вершин больше либо равно 9.
    Так, например, при размере графа 11, муравьиный алгоритм работает быстрее чем алгоритм полного перебора в 900 раз.
    Стоит отметить, что муравьиный алгоритм не гарантирует что найденный путь будем оптимальным,
    так как он является эвристическим алгоритмом, в отличии от алгоритма полного перебора.
    